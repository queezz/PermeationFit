\documentclass[../summary.tex]{subfiles}

\begin{document}

This section is a copy from \href{https://github.com/numerical-mooc/numerical-mooc}{GitHub/numerical-moc} notebooks.

\subsection{The Crank-Nicolson Method}
The \href{https://en.wikipedia.org/wiki/Crank%E2%80%93Nicolson_method}{Crank-Nicolson method} is a well-known finite difference method for the numerical integration of the heat equation and closely related partial differential equations. We often resort to a Crank-Nicolson (CN) scheme when we integrate numerically reaction-diffusion systems in one space dimension

\begin{align*}
\frac{\partial u}{\partial t} = D \frac{\partial^2 u}{\partial x^2} + f(u),\\
\frac{\partial u}{\partial x}\Bigg|_{x = 0, L} = 0,
\end{align*}
where $u$ is our concentration variable, $x$ is the space variable, $D$ is the diffusion coefficient of $u$, $f$ is the reaction term, and $L$ is the length of our one-dimensional space domain.

Note that we use \href{http://en.wikipedia.org/wiki/Neumann_boundary_condition}{Neumann boundary conditions} and specify that the solution $u$ has zero space slope at the boundaries, effectively prohibiting entrance or exit of material at the boundaries (no-flux boundary conditions).

\subsection{Finite difference methods}

Many fantastic textbooks and tutorials have been written about finite difference methods, for instance a free textbook by \href{http://people.maths.ox.ac.uk/trefethen/pdetext.html}{Lloyd Trefethen}.
Here we describe a few basic aspects of finite difference methods. The above reaction-diffusion equation describes the time evolution of variable $u(x,t)$ in one space dimension ($u$ is a line concentration).
If we knew an analytic expression for $u(x,t)$ then we could plot $u$ in a two-dimensional coordinate system with axes $t$ and $x$.

To approximate $u(x,t)$ numerically we discretize this two-dimensional coordinate system resulting, in the simplest case, in a two-dimensional \href{http://en.wikipedia.org/wiki/Regular_grid}{regular grid}.
This picture is employed commonly when constructing finite differences methods, see for instance 
\href{http://people.maths.ox.ac.uk/trefethen/3all.pdf}{Figure 3.2.1 of Trefethen}.

Let us discretize both time and space as follows:
\begin{equation}
    t_n = n \Delta t,~ n = 0, \ldots, N-1,
\end{equation}

\begin{equation}
    x_j = j \Delta x,~ j = 0, \ldots, J-1,
\end{equation}
where $N$ and $J$ are the number of discrete time and space points in our grid respectively.
$\Delta t$ and $\Delta x$ are the time step and space step respectively and defined as follows:

\begin{equation}
    \Delta t = T / N,
\end{equation}

\begin{equation}
    \Delta x = L / J
\end{equation}
where $T$ is the point in time up to which we will integrate $u$ numerically.

Our ultimate goal is to construct a numerical method that allows us to approximate the unknonwn analytic solution $u(x,t)$
reasonably well in these discrete grid points.
That is we want construct a method that computes values $U(j \Delta x, n \Delta t)$ (note: capital $U$) so that

\begin{equation}
    U(j \Delta x, n \Delta t) \approx u(j \Delta x, n \Delta t)
\end{equation}
As a shorthand we will write $U_j^n = U(j \Delta x, n \Delta t)$ and $(j,n)$ to refer to grid point $(j \Delta x, n \Delta t)$.
Let us define $\sigma = \frac{D \Delta t}{2 \Delta x^2}$ and reorder the above approximation of our reaction-diffusion equation:
\begin{equation}
    \tcboxmath{
    -\sigma U_{j-1}^{n+1} + (1+2\sigma) U_j^{n+1} -\sigma U_{j+1}^{n+1} = \sigma U_{j-1}^n + (1-2\sigma ) U_j^n + \sigma U_{j+1}^n + \Delta t f(U_j^n).
    }
    \label{eq:CN}
\end{equation}
This equation makes sense for space indices $j = 1,\ldots,J-2$ but it does not make sense for indices $j=0$ and $j=J-1$ (on the boundaries):
\begin{equation}
    j=0:~-\sigma U_{-1}^{n+1} + (1+2\sigma) U_0^{n+1} -\sigma U_{1}^{n+1} = \sigma U_{-1}^n + (1-2\sigma) U_0^n + \sigma U_{1}^n + \Delta t f(U_0^n),
\end{equation}
\begin{equation}
    j=J-1:~-\sigma U_{J-2}^{n+1} + (1+2\sigma) U_{J-1}^{n+1} -\sigma U_{J}^{n+1} = \sigma U_{J-2}^n + (1-2\sigma) U_{J-1}^n + \sigma U_{J}^n + \Delta t f(U_{J-1}^n).
\end{equation}

The problem here is that the values $U_{-1}^n$ and $U_J^n$ lie outside our grid.
However, we can work out what these values should equal by considering our Neumann boundary condition.
Let us discretize our boundary condition at $j=0$ with the \href{http://en.wikipedia.org/wiki/Finite_difference#Forward.2C_backward.2C_and_central_differences}{backward difference} and
at $j=J-1$ with the \href{http://en.wikipedia.org/wiki/Finite_difference#Forward.2C_backward.2C_and_central_differences}{forward difference}:

\begin{equation}
    \frac{U_1^n - U_0^n}{\Delta x} = 0,
\end{equation}

\begin{equation}
    \frac{U_J^n - U_{J-1}^n}{\Delta x} = 0.
\end{equation}

These two equations make it clear that we need to amend our above numerical approximation for
$j=0$ with the identities $U_0^n = U_1^n$ and $U_0^{n+1} = U_1^{n+1}$, and
for $j=J-1$ with the identities $U_{J-1}^n = U_J^n$ and $U_{J-1}^{n+1} = U_J^{n+1}$.
Let us reinterpret our numerical approximation of the line concentration of $u$ in a fixed point in time as a vector $\mathbf{U}^n$:

\begin{equation}
    \mathbf{U}^n = 
\begin{bmatrix} U_0^n \\ \vdots \\ U_{J-1}^n \end{bmatrix}.
\end{equation}
Using this notation we can now write our above approximation for a fixed point in time, $t = n \Delta t$, compactly as a linear system:

\begin{align*}
\begin{bmatrix}
1 & 0 & 0 & 0 & 0 & \cdots & 0 & 0 & 0 & 0\\
-\sigma & 1+2\sigma & -\sigma & 0 & 0 & \cdots & 0 & 0 & 0 & 0 \\
0 & -\sigma & 1+2\sigma & -\sigma & \cdots & 0 & 0 & 0 & 0 & 0 \\
0 & 0 & \ddots & \ddots & \ddots & \ddots & 0 & 0 & 0 & 0 \\
0 & 0 & 0 & 0 & 0 & 0 & 0 & -\sigma & 1+2\sigma & -\sigma \\
0 & 0 & 0 & 0 & 0 & 0 & 0 & 0 & 0 & 1
\end{bmatrix}
\begin{bmatrix}
U_0^{n+1} \\
U_1^{n+1} \\
U_2^{n+1} \\
\vdots \\
U_{J-2}^{n+1} \\
U_{J-1}^{n+1}
\end{bmatrix} 
= \\
\begin{bmatrix}
0 & 0 & 0 & 0 & 0 & \cdots & 0 & 0 & 0 & 0\\
\sigma & 1-2\sigma & \sigma & 0 & 0 & \cdots & 0 & 0 & 0 & 0 \\
0 & \sigma & 1-2\sigma & \sigma & \cdots & 0 & 0 & 0 & 0 & 0 \\
0 & 0 & \ddots & \ddots & \ddots & \ddots & 0 & 0 & 0 & 0 \\
0 & 0 & 0 & 0 & 0 & 0 & 0 & \sigma & 1-2\sigma & \sigma \\
0 & 0 & 0 & 0 & 0 & 0 & 0 & 0 & 0 & 0
\end{bmatrix}
\begin{bmatrix}
U_0^{n} \\
U_1^{n} \\
U_2^{n} \\
\vdots \\
U_{J-2}^{n} \\
U_{J-1}^{n}
\end{bmatrix} +
\begin{bmatrix}
f(U_1^n) \\
0 \\
0 \\
\vdots \\
0 \\
g(U_{J-2}^n)
\end{bmatrix}.
\end{align*}

Note that since our numerical integration starts with a well-defined initial condition at $n=0$, $\mathbf{U}^0$, the
vector $\mathbf{U}^{n+1}$ on the left-hand side is the only unknown in this system of linear equations.
Thus, to integrate numerically our reaction-diffusion system from time point $n$ to $n+1$ we need to solve numerically for vector $\mathbf{U}^{n+1}$.
Let us call the matrix on the left-hand side $A$, the one on the right-hand side $B$, and the vector on the right-hand side $\mathbf{f}^n$.
Using this notation we can write the above system as

\begin{equation}
    A \mathbf{U}^{n+1} = B \mathbf{U}^n + f^n.
\end{equation}
In this linear equation, matrices $A$ and $B$ are defined by our problem: we need to specify these matrices once for our
problem and incorporate our boundary conditions in them.
Vector $\mathbf{f}^n$ is a function of $\mathbf{U}^n$ and so needs to be reevaluated in every time point $n$.
We also need to carry out one matrix-vector multiplication every time point, $B \mathbf{U}^n$, and
one vector-vector addition, $B \mathbf{U}^n + f^n$.
The most expensive numerical operation is inversion of matrix $A$ to solve for $\mathbf{U}^{n+1}$, however we may
get away with doing this only once and store the inverse of $A$ as $A^{-1}$:

\begin{equation}
    \mathbf{U}^{n+1} = A^{-1} \left( B \mathbf{U}^n + f^n \right).
\end{equation}

\end{document}